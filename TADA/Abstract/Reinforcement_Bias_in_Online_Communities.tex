\documentclass[a4paper, 12pt]{article} %
\usepackage{graphicx,amssymb} %
\usepackage{url}			       % For \url
\usepackage{xcolor}
\usepackage[left=1cm, top=2cm, bottom = 2cm, right=1cm, nohead, nofoot]{geometry}
\usepackage{hyperref}

%\textwidth=15cm \hoffset=-1.0cm %
%\textheight=25cm \voffset=-1.5cm %

\pagestyle{empty} %

\date{} %

\def\keywords#1{\begin{center}{\bf Keywords}\\{#1}\end{center}} %

% Please, do not change any of the above lines

\tolerance=1
\emergencystretch=\maxdimen
\hyphenpenalty=10000
\hbadness=10000

\begin{document}

% Type down your paper title
\title{Reinforcement Bias in Online Communities}

\vspace{0.5cm}
% Authors
\author{ Cason Konzer \\ %
       University of Michigan - Flint \\ % Affiliation 1
       % Add authors and affiliation as needed 
       \textit{\color{violet}
       \href{mailto:casonk@umich.edu}{casonk@umich.edu}} % Only one corresponding e-mail
       }

\maketitle

\thispagestyle{empty}

% The abstract
%\vspace{2.5cm}

\begin{abstract}
\vspace{0.5cm}
% \default

    In today’s social media society there is an increasing presence of online communities and group information exchange. 
Through our community analysis tooling, we aim to address the evolution of radicalized communities, such as radically political, pseudoscience, or conspiracy theory based. 
We are concerned with intercommunity communication and recommendations. 
Research today has investigated the echo-chamber phenomenon of radicalized communities that thrive on confirmation bias. 
We would like to address reinforcement bias between similar communities. 
Do echo-chambers span across multiple communities? 
If so, is there a framework or hierarchy of linked ideologies. 
In recent news, the narrative of misinformation within the media has become highly relevant. 
Many actors have made claims taken up by the public that were either lacking evidence or outright wrong. 
Research questions to ponder are: do users within radical communities suggest other radical communities as reference or recommendation? 
Are there trends present within highly active users in radical communities? 
Is there a motive for users frequently posting misinformation? \\

    Our current project utilizes a public API based data warehouse for Reddit posts, pushshift.io. 
We have developed python scripting to automate data collection for communities based on time frame and post type. 
To address our research questions, we have scripted authorship analysis and reference analysis by community. 
From the tooling we can address those most active users within communities, and those communities most referenced. 
We have verified echo-chamber activity such that the most referenced community is itself. 
Within the communities we have researched $[$ AntiVax, FlatEarth, ConspiracyTheories, \& TrumpVirus $]$, 
we have found Trump to be the most referenced community on political actors, at approximately 2 times the occurrence of current president Biden. 
Additionally, we have found 1 in 602 posts to reference the radical group Q-anon. 
As work continues, we are interested in employing sentiment analysis within post titles and bodies, such as to extract the motive behind communal referencing. 
Other aspects of research include radical community moderation and cross-discussion.


\vspace{5.5cm}

% \color{violet}
% \textbf{Resource Management} : 
% \textit{Monitoring and Modeling Effects of Aquatic Barriers on River Ecosystems}
% \color{black}

\end{abstract}

\keywords{Reinforcement Bias, Social Media} % Write down at least 3 Keywords

% \section{Introduction}

\end{document}